
%%% Preamble
\documentclass[paper=a4, fontsize=11pt]{scrartcl}
\usepackage[T1]{fontenc}
%\usepackage{fourier}

\usepackage[english]{babel}															% English language/hyphenation
\usepackage[protrusion=true,expansion=true]{microtype}	
\usepackage{amsmath,amsfonts,amsthm} % Math packages
\usepackage[pdftex]{graphicx}	
\usepackage{url}


%%% Custom sectioning
%\usepackage{sectsty}
%\allsectionsfont{\centering \normalfont\scshape}


%%% Custom headers/footers (fancyhdr package)
\usepackage{fancyhdr}
\pagestyle{fancyplain}
\fancyhead{}											% No page header
\fancyfoot[L]{}											% Empty 
\fancyfoot[C]{}											% Empty
\fancyfoot[R]{\thepage}									% Pagenumbering
\renewcommand{\headrulewidth}{0pt}						% Remove header underlines
\renewcommand{\footrulewidth}{0pt}						% Remove footer underlines
\setlength{\headheight}{13.6pt}


%%% Equation and float numbering
\numberwithin{equation}{section}		% Equationnumbering: section.eq#
\numberwithin{figure}{section}			% Figurenumbering: section.fig#
\numberwithin{table}{section}			% Tablenumbering: section.tab#


%%% Maketitle metadata
\newcommand{\horrule}[1]{\rule{\linewidth}{#1}} 	% Horizontal rule

\title{
	%\vspace{-1in} 	
	\usefont{OT1}{bch}{b}{n}
	\normalfont \normalsize \textsc{University of Central Arkansas} \\ [25pt]
	\horrule{0.5pt} \\[0.4cm]
	\large Independent Study Report CRN: 31231 \\
	\huge Caffeinated Deep Learning \\
	\horrule{2pt} \\[0.5cm]
}
\author{
	\normalfont 			\normalsize
	Abay Bektursun\\[-3pt]	\normalsize
	Fall 2017\\[-3pt]	\normalsize
}
\date{}



%%% Begin document
\begin{document}
	
	\maketitle
	\section{Fundamentals of Deep Learning}
		In this section I will shed some light onto the mathematical and computational concepts needed for understanding the basics of Deep Learning or any analytical methods for that matter. Most of the mathematical notation utilized here is same or similar to the one used in the famous "Deep Learning Book" [TODO: .bib]\par
	Test 
	
	
	\subsection{Numerical Computation}
	Lorem ipsum dolor sit amet, consectetuer adipiscing elit. 
	
	Aenean commodo ligula eget dolor. Aenean massa. Cum sociis natoque penatibus et magnis dis parturient montes, nascetur ridiculus mus. Donec quam felis, ultricies nec, pellentesque eu, pretium quis, sem.
	
	\subsection{Relevant Optimization Methods}
	\subsubsection{First-Order Optimization Algorithms}
	\subsubsection{Second-order Optimization Algorithms}
	
	\subsection{Neural Networks}
	Lorem ipsum dolor sit amet, consectetuer adipiscing elit. Aenean commodo ligula eget dolor. Aenean massa. Cum sociis natoque penatibus et magnis dis parturient montes, nascetur ridiculus mus. Donec quam felis, ultricies nec, pellentesque eu, pretium quis, sem. Nulla consequat massa quis enim. 
	
	\section{Introduction to Caffe}
	\section{Training Convolutional Neural Network on Caffe}
	\section{Testing and Tuning Hyperparameters}
	\section{Analysis of the Results and Further Investigation of Conv Net Architectures}
	\section{Classification of Skin Cancer Using GPU Accelerated [Architecture] on Caffe}
	\section{Analysis of the Skin Cancer Classification Results}
	
	
	\section{Lists}
	
	\subsection{Example for list (3*itemize)}
	\begin{itemize}
		\item First item in a list 
		\begin{itemize}
			\item First item in a list 
			\begin{itemize}
				\item First item in a list 
				\item Second item in a list 
			\end{itemize}
			\item Second item in a list 
		\end{itemize}
		\item Second item in a list 
	\end{itemize}
	
	\subsection{Example for list (enumerate)}
	\begin{enumerate}
		\item First item in a list 
		\item Second item in a list 
		\item Third item in a list
	\end{enumerate}
	%%% End document
	
	\begin{align} 
	\begin{split}
	(x+y)^3 	&= (x+y)^2(x+y)\\
	&=(x^2+2xy+y^2)(x+y)\\
	&=(x^3+2x^2y+xy^2) + (x^2y+2xy^2+y^3)\\
	&=x^3+3x^2y+3xy^2+y^3
	\end{split}					
	\end{align}
	
	\begin{align}
	A = 
	\begin{bmatrix}
	A_{11} & A_{21} \\
	A_{21} & A_{22}
	\end{bmatrix}
	\end{align}
	

\bibliographystyle{plain}
\bibliography{ref}


\end{document}
